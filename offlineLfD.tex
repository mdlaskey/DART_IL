%%%%%%%%%%%%%%%%%%%%%%%%%%%%%%%%%%%%%%%%%%%%%%%%%%%%%%%%%%%%%%%%%%%%%%%%%%%%%%%%
%2345678901234567890123456789012345678901234567890123456789012345678901234567890
%        1         2         3         4         5         6         7         8

%\documentclass[journal,transmag]{IEEEtran}% Comment this line out if you need a4paper

\documentclass[10pt, conference]{ieeeconf}      % Use this line for a4 paper


\IEEEoverridecommandlockouts                              % This command is only needed if 
                                                          % you want to use the \thanks command

%\overrideIEEEmargins                                      % Needed to meet printer requirements.

% See the \addtolength command later in the file to balance the column lengths
% on the last page of the document

% The following packages can be found on http:\\www.ctan.org
%\usepackage{graphics} % for pdf, bitmapped graphics files
%\usepackage{epsfig} % for postscript graphics files
%\usepackage{mathptmx} % assumes new font selection scheme installed
%\usepackage{times} % assumes new font selection scheme installed
%\usepackage{amsmath} % assumes amsmath package installed
%\usepackage{amssymb}  % assumes amsmath package installed

\newtheorem{theorem}{Theorem}[section]
\newtheorem{lemma}[theorem]{Lemma}
\newtheorem{proposition}[theorem]{Proposition}
\newtheorem{corollary}[theorem]{Corollary}
\usepackage[ruled,vlined]{algorithm2e}
\usepackage{url}
\newenvironment{definition}[1][Definition]{\begin{trivlist}
\item[\hskip \labelsep {\bfseries #1}]}{\end{trivlist}}

\newcommand{\qed}{\nobreak \ifvmode \relax \else
      \ifdim\lastskip<1.5em \hskip-\lastskip
      \hskip1.5em plus0em minus0.5em \fi \nobreak
      \vrule height0.75em width0.5em depth0.25em\fi}

\def\lc{\left\lfloor}   
\def\rc{\right\rfloor}

\usepackage{amsmath,amssymb}

\usepackage{tabularx}
\usepackage{tikz,hyperref,graphicx,units}
\usepackage{subfigure}
\usepackage{benktools}
\usepackage{bbm}
\renewcommand{\baselinestretch}{.5}

\usepackage{caption}
\usepackage{epstopdf}
\renewcommand{\captionfont}{\footnotesize}
\usepackage{sidecap,wrapfig}
\usepackage[ruled,vlined]{algorithm2e}
\DeclareMathOperator*{\argmin}{arg\,min}
\DeclareMathOperator*{\argmax}{arg\,max}
\newcommand{\abs}[1]{\lvert#1\rvert} 
\newcommand{\norm}[1]{\lVert#1\rVert}
%\newcommand{\suchthat}{\mid}
\newcommand{\suchthat}{\ \big|\ }
\newcommand{\ba}{\mathbf{a}}
\newcommand{\bb}{\mathbf{b}}
\newcommand{\bc}{\mathbf{c}}
\newcommand{\bd}{\mathbf{d}}
\newcommand{\bg}{\mathbf{g}}
\newcommand{\bj}{\mathbf{j}}
\newcommand{\bn}{\mathbf{n}}
\newcommand{\bp}{\mathbf{p}}
\newcommand{\bw}{\mathbf{w}}
\newcommand{\bt}{\mathbf{t}}
\newcommand{\bu}{\mathbf{u}}
\newcommand{\by}{\mathbf{y}}
\newcommand{\bx}{\mathbf{x}}
\newcommand{\bz}{\mathbf{z}}
\newcommand{\bbf}{\mathbf{f}}
\newcommand{\bzero}{\mathbf{0}}
\newcommand{\bG}{\mathbf{G}}
\newcommand{\bA}{\mathbf{A}}
\newcommand{\bW}{\mathbf{W}}
\newcommand{\bX}{\mathbf{X}}
\newcommand{\mX}{\mathcal{X}}
\newcommand{\mD}{\mathcal{D}}
\newcommand{\mG}{\mathcal{G}}
\newcommand{\mN}{\mathcal{N}}
\newcommand{\mW}{\mathcal{W}}
\newcommand{\mF}{\mathcal{F}}
\newcommand{\bZ}{\mathbf{Z}}
\newcommand{\mR}{\mathcal{R}}

\newcommand{\bfc}{W}
\newcommand{\Qinf}{Q_{\infty}}
\newcommand{\st}[1]{_\text{#1}}
\newcommand{\rres}{r\st{res}}
\newcommand{\pos}[1]{(#1)^+}
\newcommand{\depth}{\operatorname{depth}}
\newcommand{\dist}{\operatorname{dist}}
\newcommand{\convhull}{\operatorname{ConvexHull}}
\newcommand{\minksum}{\operatorname{MinkowskiSum}}

\newcommand{\specialcell}[2][c]{ \begin{tabular}[#1]{@{}c@{}}#2\end{tabular}}
\newcommand{\acro}{SHIV}
\newcommand\independent{\protect\mathpalette{\protect\independenT}{\perp}}
\def\independenT#1#2{\mathrel{\rlap{$#1#2$}\mkern2mu{#1#2}}}

\newcolumntype{L}[1]{>{\RaggedRight\hspace{0pt}}p{#1}}
\newcolumntype{R}[1]{>{\RaggedLeft\hspace{0pt}}p{#1}}


\newboolean{include-notes}
\setboolean{include-notes}{true}
\newcommand{\adnote}[1]{\ifthenelse{ \boolean{include-notes}}%
 {\textcolor{blue}{\textbf{AD: #1}}}{}}
 
 \newcommand{\sknote}[1]{\ifthenelse{ \boolean{include-notes}}%
 {\textcolor{blue}{\textbf{SK: #1}}}{}}
 
  \newcommand{\mlnote}[1]{\ifthenelse{ \boolean{include-notes}}%
 {\textcolor{purple}{\textbf{ML: #1}}}{}}
 
 \newcommand{\jmnote}[1]{\ifthenelse{ \boolean{include-notes}}%
 {\textcolor{orange}{\textbf{JM: #1}}}{}}

\renewcommand{\baselinestretch}{.95}
\usepackage{times}
\usepackage{microtype}
%\title{Iterative Imitation Learning with Reduced Human Supervision [v11]}
%\title{SHIV:  Reducing Human Supervision for Robot adaptive Learning [v11]}

\title{An Analysis of Adaptivity
 \\in  Robotic Learning from Demonstrations}



\author{Michael Laskey, Caleb Chuck, Jonathan Lee, Jeff Mahler,\\ Sanjay Krishnan, Kevin Jamieson, Anca Dragan, Ken Goldberg}
\begin{document}




\maketitle
\thispagestyle{empty}
\pagestyle{empty}


%%%%%%%%%%%%%%%%%%%%%%%%%%%%%%%%%%%%%%%%%%%%%%%%%%%%%%%%%%%%%%%%%%%%%%%%%%%%%%%%




%%%%%%%%%%%%%%%%%%%%%%%%%%%%%%%%%%%%%%%%%%%%%%%%%%%%%%%%%%%%%%%%%%%%%%%%%%%%%%%%

\begin{abstract}
Learning from demonstration algorithms are either passive or adaptive . It has been shown in robotics, adaptive learning from demonstrations (e.g. DAgger) perform better than passive, which has been attributed to training on the distribution induced by the robot's policy. We argue that this success is also because the supervisor's policy  is not in the same function class of the robot's policy, or not realizable. However with recent advances in boosting and deep learning, situations where realizability is possible could be more prevalent.  We show theoretically, when the supervisor's policy is realizable on their distribution, adaptive learning from demonstration techniques can fail to converge to the true supervisor's policy, but passive is able to. We also demonstrate empirically in grid world simulation  and via human trials on a robot, that adaptive techniques can require more data even when convergence is possible. \mlnote{updating abstract tonight}
 \end{abstract}


\section{Introduction} 
In model-free robot Learning from Demonstration (LfD), a robot learns to perform a task, such as driving or grasping an object in a cluttered environment, from examples provided by an expert, usually a human. Learning from demonstration has been applied to a large number of robotic tasks, including helicopter maneuvering~\cite{abbeel2007application}, car parking~\cite{abbeel2008apprenticeship} , robotic surgery~\cite{van2010superhuman,laskeyshiv} and robotic manipulation~\cite{laskeyrobot}. 

In the supervised learning approach to LfD, a robot's policy , or a mapping from the state space to controls, is learned via regression or classification. The passive approach to LfD is to collect expert demonstrations on the states the expert is likely to visit. However, the conventional wisdom is that due to error in learning the robot's policy will not match the expert and cause it to visit different states when performing the task. A common trend in LfD, is adaptivity, which is to have the expert retro-actively provide feedback (or the correct control signal) to the robot on the states that it visits~\cite{ross2010efficient,ross2010reduction,laskeyrobot,laskeyshiv,he2012imitation}.



Error in learning though can occur due to two reasons: 1) is the model, or robot's policy, is not capable of representing a the expert's expected policy or 2) there is currently not enough demonstrations to recover the expected expert's policy. If the first reason occurs than the robot will never recover the expected expert's policy even in the limit of infinite data. Adaptivity may be useful because the robot will always visit different states than the expert. 

However, in second case where not enough data has been collected. A question exists should it be better to collect data from the distribution of the current robot's policy or that of the expert's. Our insight, it that  when the robot's policy is capable of learning the expected expert's policy the robot will want to collect data from solely the expert's distribution. 


This insights stems from two possible problems: 1) is that adaptive techniques force the robot into areas of the state space that the expert would never visit and could require the robot to try and learn a more complex policy 2) is that humans can have trouble retroactively providing feedback without knowledge of the outcome of the control. 

\begin{figure}
\center
\includegraphics[width=0.5\textwidth]{f_figs/teaser.eps}
\caption{
    \footnotesize
A) An illustration of  the expert's policy, $\pi_{\theta^*}$ being contained in the robot's policy class $\Theta$. We argue that in the case on the right, passive LfD can be advantageous compare to adaptive. B) Samples taken from a robot trying to reach the green goal state. The adaptive approach(Teal) causes the robot to visit states the passive (Orange) does not need to. C) A person from our pilot study using an Xbox Controller to provide a demonstration to the robot on how to singulate an object from the pile. D) In the pilot study,  passive (left) always had visit the states the expert did, but adaptive (right) lead to states where the expert had to provide more complex commands such as teaching the robot to go backwards.}
\vspace*{-20pt}
\label{fig:teaserl}
\end{figure}


 We tested the occurrence of the these problem via a pilot study with 10 roboticists at UC Berkeley, who are asked to use both DAgger and Passive Lfd to train a Zymark robot to singulate (or seperate an object) from a pile. We observed a statistically significant gap in the average performance of the policies trained by the two approaches. In our  post analysis we examined how well people could match their controls in retro-active feedback compared to tele-operated. Furthermore, we examined a held out test set's surrogate loss to illustrate how well the policy learned is able to generalize. A more complex policy should exhibit less generalization behavior. 

Additionally, we contribute three theoretical contributions. First, is that we show adaptive algorithms despite having the same function class as the expert, on the expert's distribution, can with non-zero probability not recover the expert policy with infinite demonstrations. Thus, illustrating a key problem that could occur while choosing between adaptive versus passive techniques. 

We furthermore contribute a new analysis for the error accumulation in passive Lfd, that demonstrates for squared euclidean losses a rate of $O(T\sqrt{T})$, where $T$ is the time horizon, is achievable in contrast to the $O(T^2)$ rate suggested by Ross et al.~\cite{ross2010efficient}. We finally provide a data-dependent analysis of this rate for convex policy classes (i.e. linear svm, kernelized regression) using Rademacher Complexity, which shows that $T$ is not only important variable in analyzing policy error, but also function class size and number of demonstrations play a key role. 

We  provide experiments in a gridworld domain over 100 randomly generated environments demonstrating the effect of growing the robot's policy function class size. Result suggests that the performance difference between adaptive and passive is only seen when expert's policy is not contained in the set. We then illustrate a 2D point mass example, in which despite having the same function class as the expert, adaptivity fails to converge to the expert's policy. Finally, we test both DAgger and passive Lfd for learning a visuo-motor neural network policy for singulating objects from a pile on a Zymark Robot platform across 10 human subjects. We found the passive LfD is able to achieve a $21 \%$ increase in the probability of success with the same amount of data.




\section{Related Work}
Below we summarize related work in passive and adaptive LfD and there theoretical insights. 

\noindent \textbf {Passive LfD}
Pormeleau et al. used passive Lfd to train a neural network to drive a car on the highway via demonstrations provided by the expert. To reduce the number of demonstrations needed to perform a task, they synthetically generated  images of the road and subsequent labels~\cite{pomerleau1989alvinn}. A similar idea, was recently proposed by ~\cite{NVIDEA}, but used a convolutional neural architecture. 

Ross et al. examined the passive Lfd in a theoretical setting and derived that in the worst case the error from this approach can go quadratic in the time  horizon, $T$~\cite{ross2010efficient}. The intuition behind this analysis is that if the distribution induced by the robot's policy is different than the supervisor's, the robot could incur maximum error. 

In our analysis, we show that a rate of $T\sqrt{T}$ is achievable. However in contrast to Ross et al,  we also show how function class complexity and the number of demonstrations effect this bound. Data dependent results can help provide better insight into the error incurred in the finite sample domains. 

\noindent \textbf{Adaptive LfD}
Adaptive LfD has recently been used in numerous examples of model-free learning from high dimensional state representations, such as images. Successful robotic examples of adaptive LfD with an expert supervisor include applications to flying a quad-copter through a forest where the input is only image data taken from an on board sensor~\cite{ross2013learning}.

 Recently, Laskey et al. applied adaptive LfD to manipulation tasks such as surgical needle insertion \cite{laskeyshiv} and robotic de-cluttering, where a robot is given image data of a table with a variety of objects on it and must learn to push the obstacle objects aside to grasp a goal \cite{laskeyrobot}. Other succesful examples have been teaching a robotic wheel chair navigate to goal positions in the presence of obstacles and teaching a robot to follow verbal instructions to navigate across an office building \cite{kim2013maximum, duvallet2013imitation}. 

Algorithmic extensions to adaptive LfD have also been recently made, such as  forcing the supervisor to provide controls more similar to the robot's policy which allows for easier to learn policies~\cite{he2012imitation}. Furthermore, Kim et al. proposed only query the supervisor in states that the robot is uncertain~\cite{kim2013maximum} about and Laskey et al. extended this to high dimensional states~\cite{laskeyshiv}. Finally, Laskey et al. looked at using a hierarchy (in terms of quality) of supervisors to reduce burden on the expert supervisor~\cite{laskeyrobot}.

All these approaches build theoretically upon the online optimization analysis that Ross et al. proposed~\cite{ross2010reduction}. In online optimization an adversary presents a learner's policy with some cost function and the learner then chooses an action and receives a loss. In the LfD context, the learner chooses a policy and the adversary always select the expected disagreement with respect to the supervisor on the distribution induced by the policy~\cite{shalev2011online}. DAgger specifically can be modeled as a Follow-The-Leader algorithm, because it picks the best policy over all previous seen demonstrations~\cite{ross2010reduction}.

In the online optimization algorithm a bound for the error on the robot's policy can be obtained that is linear in $T$ for stongly convex losses (i.e. heavly regularized linear or kernelized regression). However as we show the regret bound does not imply convergance to the supervisor is possible, even when the robot's function class contains the expected supervisor's policy. 


\section{Problem Statement and Background}\label{sec:PS}
The goal of this work is to learn a policy that matches that of the supervisor's on a specified task that demonstrations are collected on. 

\noindent\textbf{Modeling Choices and Assumptions}  We model the system dynamics as Markovian, stochastic, and stationary. Stationary dynamics occur when, given a state and a control, the probability of the next state does not change over time. 

We model the initial state as sampled from a distribution over the state space.
We assume a known state space and set of controls. We also assume access to a robot or simulator, such that we  can sample from the state sequences induced by a sequence of controls.   Lastly, we assume access to a supervisor who can, given a state, provide a control signal label. We additionally assume the supervisor can be noisy and imperfect, noting that the robot cannot surpass the level of performance of the supervisor. 



\noindent\textbf{Policies and State Densities.}
Following conventions from control theory, we denote by $\mathcal{X}$ the set consisting of observable states for a robot task, consisting, for example, of 
high-dimensional vectors corresponding to images from a camera, or robot joint angles and object poses in the environment.
We furthermore consider a set $\mathcal{U}$ of allowable control inputs for the robot, which can be discrete or
continuous. We model dynamics as Markovian, such that the probability of state $\mathbf{x_{t+1}}\in
\mathcal{X}$ can be determined from the previous state $\mathbf{x}_t\in\mathcal{X}$ and control input $\mathbf{u}_t\in
\mathcal{U}$: 
$$p(\bx_{t+1}|\bu_{t},\bx_{t}, \ldots, \bu_{0}, \bx_{0})=p(\bx_{t+1}|\bu_{t}, \bx_t)$$
We assume a probability density over initial states $p(\bx_0)$.


%We denote the probability density over the initial state also by $p:\mathcal{X}\to \mathbb{R}$. 

A trajectory $\hat{\tau}$ is a finite sequence of $T+1$ pairs of states visited and corresponding
control inputs at these states, $\hat{\tau} = (\mathbf{x}_0,\mathbf{u}_0, ...., \mathbf{x}_T,\mathbf{u}_T)$, where $\bx_t\in \mathcal{X}$
and $\bu_t\in \mathcal{U}$ for $t\in \{0, \ldots, T\}$ and some $T\in \mathbb{N}$.  
For a given trajectory $\hat{\tau}$ as above, we denote by ${\tau}$ the corresponding trajectory in state space,
${\tau} = (\bx_0,....,\bx_T)$.


A policy is a measurable function $\pi: \mathcal{X} \to \mathcal{U}$ from states to control inputs. 
We consider a set of policies $\pi_{\theta}:\mathcal{X}\to \mathcal{U}$ parameterized by some $\theta\in \Theta$. Any such policy $\pi_{\theta}$ in an environment with probabilistic initial state density and Markovian dynamics
induces a density on probability measure over the set of  trajectories of length $T+1$: $$p(\tau | \theta)=
p(\bx_0)\prod_{i=0}^{T-1}p(\bx_{t+1}|\bu_t,\bx_t)p(\bu_t|\bx_t,\theta)$$


While we do not assume knowledge of the distributions corresponding to: $p(\bx_{t+1}|\bx_t,\bu_t)$, $p(\bx_0)$ or $p(\bx_t|
\theta)$, we assume that we have a stochastic real robot or a simulator such that for any state
$\bx_t$ and control $\bu_t$, we can sample the $\bx_{t+1}$ from the density $p(\bx_{t+1}|\bu_t,\bx_t)$. 
Therefore, when 'rolling out' trajectories under a policy
$\pi_{\theta}$, we utilize the robot or a simulator to sample the resulting stochastic trajectories rather than
estimating $p(\bx|\theta)$ itself.


\noindent\textbf{Objective.} The objective of  policy learning is to find a policy that minimizes some known cost function $C(\hat{\tau}) = \sum^T_{t=1} c(\bx_t,\bu_t)$ of a trajectory $\hat{\tau}$. The cost $c:\mathcal{X}\times \mathcal{U}\to \mathbb{R}$ is typically user defined and task specific. 
For example, in the task of inserting a peg into a hole, a function on distance between the peg's current and desired final state is used \cite{levine2015end}.  


In our problem, we do not have access to the cost function itself. Instead, we only have access to 
a supervisor, $\pi_{\theta^*}$, where $\theta^*$ may not be contained in $\Theta$. A supervisor is chosen that can achieve a desired level of performance on the task. The supervisor provides the robot an initial set
of $N$   demonstration trajectories $\lbrace \tilde{\tau}^1,...,\tilde{\tau}^N \rbrace$. 
which are the result of the supervisor applying this policy. This induces a training data set $\mathcal{D}$ of all state-control input pairs from the demonstrated trajectories.

We are interested in determining what parameter $\theta$ generates the sample demonstrations from the supervisors policy. 

\noindent \textbf{Passive LfD} Passive Lfd frames this  question as maximizing the conditional likelihood of the sample demonstrations conditioned on a given parameter $\theta$. 

$$\underset{\theta}{\mbox{max}} \prod^N_{n=1} p(\bx_0,n) \prod^T_{t=1} p(\bx_{t+1,n}|\bx_{t,n},\bu_{t,n})p(\bu_t|\bx_t,\theta)$$

In solving this optimization it is common to optimize the conditional log-likelihood, which maintains the same solution but breaks up the product terms into sums. 

\begin{equation}\label{eq:m_likeli_obj}
\underset{\theta}{\mbox{max}} \sum^N_{n=1}\sum^T_{t=1}\mbox{log }p(\bu_t|\bx_t,\theta)
\end{equation}


We note that the dynamics and initial state distributions are dropped in this objective because they are conditionally independent of $\theta$, once the controls are observed. 

 Traditionally maximizing the probability of a given control observed from the supervisor, has been viewed as minimization of a surrogate loss to denote a difference between the true cost function~\cite{ross2010reduction,ross2010efficient}. We will refer the to function $l : \mathcal{U} \times \mathcal{U} \rightarrow \mathbb{R}$ as the surrogate loss through out this paper. The surrogate loss can either be an indicator function as in classification or a continuous measure on the sufficient statistics of $p(\bu|\bx,\theta)$.  This rewrites the objective as follows: 

\begin{equation}\label{eq:main_obj}
\theta^N = \underset{\theta}{\mbox{argmin}} \sum^N_{n=1}\sum^T_{t=1} l(\bu_{n,t}, \pi_{\theta} (\bx_{n,t})).
\end{equation}


\noindent \textbf{Adaptive LfD}
Due to model mismatch (e.g. not being realizable), or a limited amount of data, solving Eq. \ref{eq:main_obj} may not lead to $\hat{\theta} \neq \theta^*$.  Thus leading to a mismeasure in the distribution trained on and tested on, since the robot visits states from the distribution $p(\tau|\hat{\theta})$ and not the supervisors, $p(\tau|\theta^*)$.  Prior work has proposed an adaptive solution ~\cite{ross2010reduction} that attempts to solves this problem by aggregating data to gather data on the distribution induced by the robot's policy.

Instead of  minimizing the surrogate loss, in Eq. \ref{eq:main_obj},  DAgger attempts to find the distribution the final policy will converge to. Thus reducing surrogate loss in places the robot is likely to visit.
DAgger~\cite{ross2010reduction} attempts this by iterating two steps: 1)
computing the policy parameter $\theta$ using the training data $\mathcal{D}$ thus far, and 2) by executing the policy
induced by the current $\theta$, and asking for labels for the encountered states. 
 
\subsubsection{Step 1}
The first step of any iteration $k$ is to compute a $\hat{\theta}_k$ that minimizes surrogate loss on the current dataset $\mathcal{D}_k=\{(x_i,u_i)|i\in\{1,\ldots,M\}\}$ of demonstrated state-control pairs (initially just the set $\mathcal{D}$ of initial trajectory demonstrations):

 \vspace{-1ex}
\begin{align}\label{eq:super_objj}
\theta_{k} = \underset{\theta}{\argmin} \: \sum_{i=1}^{M} \sum_{t=1}^T  l(\pi_{\theta}(\bx_{i,t}),\bu_{i,t}).
\end{align}

This can be observed as minimizing the empirical risk on the aggregate dataset of all examples seen so far, note that equal weight is giving to each example regardless of high likely they under the current policy~\cite{scholkopf2002learning}. 
 

 \subsubsection{Step 2}
The second step at iteration $k$, DAgger rolls out the current policy, $\pi_{\theta_{k}}$, to sample states that are likely under $p(\tau|\theta_{k})$.  For every state visited, DAgger requests the supervisor to provide the appropriate control/label. Formally, for a given sampled trajectory  $\hat{\tau} = (\bx_0,\bu_0,...,\bx_T,\bu_T )$, the supervisor provides labels $\tilde{\bu}_t$, where $\tilde{\bu}_t \sim \tilde{\pi}(\bx_t) + \epsilon$, where $\epsilon$ is a  zero mean noise term, for $t\in \{0, \ldots, T\}$.
The states and labeled controls are then aggregated into the next data set of demonstrations $\mathcal{D}_{k+1}$:
$$D_{k+1}=\mathcal{D}_k \cup \{(\bx_t,\tilde{\bu_t})|t\in\{0,\ldots,T\}\} $$

Steps 1 and 2 are repeated for $K$ iterations or until the robot has achieved sufficient performance on the
task\footnote{In the original DAgger the policy rollout was stochastically mixed with the supervisor, thus with
    probability $\beta$ it would either take the supervisor's action or the robots. The use of this stochastically mix
    policy was for theoretical analysis and in practice, it was recommended to set $\beta = 0$ to avoid biasing the
sampling~\cite{NIPS2014_5421,ross2010reduction}}.


 

\section{Theoretical Analysis}
In this section we will first show how adaptive algorithms can cause the robot's policy to not converge. Then we present a new analysis for passive LfD, which shows a better scaling rate for the time horizon than suggested in prior literature.  Finally, we use Rademacher Complexity to illustrate how function class and number of demonstrations affect the worst case error. 

For this section are interested in the total cost along a trajectory with respect to a policy, $\pi_{\theta}$, which is defined as $J(\theta) = \sum^T_{t=1} l(\pi_{\theta}(\bx_{t}),\pi_{\theta^*}(\bx_{t}))$. 

\subsection{Algorithm Convergence}

Convergence of an LfD algorithm can be defined in a number of ways, for example one way could be the element-wise condition $\theta^* = \hat{\theta}$. However, this is potentially a stronger condition than required for the performance of a robot to match that of an expert.  This is because the robot only needs to match the expert on the distribution induced by the expert (i.e. if the expert never visits a state the robot should not be forced to match the expert there). 

Thus, we define convergence as given an infinite amount of data the following must be true: 

$$E_{p(\tau|\theta^*)}J(\theta^*) = E_{p(\tau|\theta^*)}J(\hat{\theta}) $$

This statement implies that the policy achieves the same surrogate loss of the expert's policy on the distribution of states the expert visits. Thus, agrees with the expert in all states the expert would visit.  The function class $\Theta$ and choice of learning algorithm that achieves this condition will be defined as expected realizable. 

We will now show that it is possible for a given policy class, $\Theta$, to achieve have this condition when using passive LfD, but fail to when using an adaptive LfD. \\


\begin{figure}
\centering
\includegraphics{f_figs/counter_exmp.eps}
\caption{
    \footnotesize
A binary decision tree, where a robot is being taught by a supervisor to descend down and achieve maximum cumulative reward, which is shown via the numbers on each branch. Each node has the action the supervisor would select, either left or right, $\lbrace L, R \rbrace$. The optimal policy, colored in green, is to select left, $L$, at each node.}
\vspace*{-20pt}
\label{fig:c_ex}
\end{figure}


\begin{theorem}
Given a policy class, $\Theta$, that is expected realizable under passive LfD. There exists environments such that under adaptive LfD with non-zero probability become not expected realizable. \\
\end{theorem}

\begin{proof}
Since, we are looking at situations where passive LfD has the condition of expected realizability, we must only demonstrate the existence of one environment where adaptive LfD fails to converge with non-zero probability. 

Consider where a robot is learning to search down a tree to maximize cumulative reward, as illustrated in Fig. \ref{fig:c_ex}. Here the robot's policy class is the finite set of only being able to choose left or right, $\Theta = \lbrace L,R \rbrace$.

The learning algorithm, which chooses how to update the policy is to choose the policy, which agrees with the expert's controls the most, or: 

$$\pi_{\theta} = \underset{\lbrace L,R \rbrace}{\mbox{argmin}} \sum^M_{m=0}\sum^T_{t=0} I(\pi_\theta(x_{m,t}),\tilde{\pi}(x_{m,t}))$$

If passive LfD is applied to this situation, the expert will always chose left, $L$, and the robot will select this action resulting in perfect imitation. However, if DAgger is used and the robots initial policy is to go right, $\pi_{\theta_0} = R$. Then the robot will repeatedly always choose right and never converge. 

Thus, convergence in regret is achievable because the best the robot could have done in hindsight would be to always choose $R$, but that does not result in a good policy. We note that if passive LfD was applied the robot would have chosen $L$ from the a single initial demonstration. 

\end{proof}

It is interesting to note that this theorem does not contradict the original theoretical anlysis of Ross et al. ~\cite{ross2010reduction}. In their analysis, DAgger is modeled as a  an online optimization problem~\cite{ross2010reduction}.  In online optimization a learner plays a game where at each iteration, it chooses a policy and receives a loss from an adversary.  In the LfD setting, the learner is the robot's policy and the adversary would be the loss on the distribution of states induced by the policy $E_{p(\tau|\theta)} J(\theta)$~\cite{shalev2011online}.

DAgger in this context is known as a Follow the Leader (FTL) algorithm. In FTL, the best policy is chosen on all previous seen losses, or the aggregate dataset in the LfD context. Their analysis shows that under the condition when the surrogate loss, $l$, is strongly convex with respect to $\theta$, their policy has a regret that converges to the best the policy could have done on all previous seen losses $\underset{\theta}{\min} \sum_{k=1}^K E_{p(\tau|\hat{\theta}_k)}J(\hat{\theta}_k)$. 

A bound in regret though though does not necessarily imply the robot will be able to generalize to the unseen data nor match the expert. It instead says the robot's error is bounded by the best it could have done in hindsight, which may be arbitrary bad as our tree example suggests. 

We do acknowledged though that these  problem can be overcome by increasing the robot policy's function class. However, a larger function class could result in more data needed to learn the policy than the passive LfD and be more susceptible to noise~\cite{kakade2009generalization}.

\subsection{Bound on Error for Passive Lfd}
In the passive LfD setting a robot is trained on the states visited by the expert. However, at run time it can encounter a different distribution of states due to not perfectly matching the expert's policy. Ross et al. showed that given a time horizon, $T$, the worst case error scales quadritically (i.e. $O(T^2)$) when executing the robot's policy~\cite{ross2010efficient}. We present a new analysis for a specific class of policies defined below that shows a rate of $O(T\sqrt{T})$, which suggests a tamer rate in the time time horizon. 

 Define the surrogate loss as the squared euclidean norm, or $l(\pi_{\theta}(\bx),\pi_{\theta^*}(\bx)) = ||\pi_{\theta}(\bx_{i,t}) - \pi_{\theta^*}(\bx_{i,t})||_2$.We assume that the controls are be bounded and thus can be normalized such that the $l \in [0,1]$.  We are interested in the situation where the supervisor and robot policy are stochastic with a Normal Distribution (i.e. $p(\bu|\pi_{\theta}(\bx)) = \mathcal{N}(\pi_\theta(\bx),\sigma I)$. A stochastic policy with a normal distribution is a common assumption for learned robotic policy ~\cite{levine2015end}. 
 

\begin{theorem}
Given a policy $\pi_{\theta^N}$, the following is true 
$$E_{p(\tau|\theta^n)} J(\theta^N) \leq \sqrt{T\frac{1}{4\sigma}E_{p(\tau|\theta^*)} J(\theta^N)}+E_{p(\tau|\theta^*)} J(\theta^N)$$\\
\end{theorem}
\begin{proof}
For convenience we will write $E_{p(\tau|\theta)} = E_{\theta}$ and $l(\theta,\bx) = l(\theta)$. 

\begin{align}
&E_{\theta^N} J(\theta^N) - E_{\theta^*} J(\theta^N) \\
&= T(\frac{1}{T}E_{\theta^N} J(\theta^N) -\frac{1}{T}E_{\theta^*} J(\theta^N)\\
&\leq  T\mbox{sup}_{\theta} |\frac{1}{T}E_{\theta_n}J(\theta) - \frac{1}{T}E_{\theta^*} J(\theta)|\\
&\leq  T| | p(\tau|\theta^N) - p(\tau|\theta^*)||_{TV}\\
&\leq T\sqrt{\frac{1}{2} D_{KL}(p(\tau|\theta^*),p(\tau|\theta^N))}
\end{align}

Line 5 bounds the expected loss over $\theta^N$ by taking the worst case value. Line 6 leverages the fact that the worst case loss is bounded by $1$ and the definition of Total Variational distance. Line 7 uses Pinsker's inequality. 


\begin{align}
&= T\sqrt{\frac{1}{2} E_{p(\theta^*)} \mbox{log} \frac{p(\tau|\theta^*)}{p(\tau|\theta^N)}}\\
&= T\sqrt{\frac{1}{2} E_{p(\theta^*)} \sum^T_{t=1}\mbox{log} \frac{p(\bu_t|\bx_t,\theta^*)}{p(\bu_t|\bx_t,\theta^N)}}\\
&= T\sqrt{\frac{1}{4\sigma} E_{p(\theta^*)} \sum^T_{t=1} ||\bu_t- \pi_{\theta^N}(\bx_t)||_2^2 - ||\bu_t- \pi_{\theta^*}(\bx_t)||_2^2}\\
&\leq T\sqrt{\frac{1}{4\sigma} E_{p(\theta^*)} \sum^T_{t=1}  ||\pi_{\theta^*}(\bx_t) - \pi_{\theta^N}(\bx_t)||_2^2}\\
&= T\sqrt{\frac{1}{4\sigma} E_{p(\theta^*)} J(\theta^N)}
\end{align}

Line 8,9 and 10 apply the definition of the KL-divergence, the markov chain and the normal distribution over $p(\bu_t|\bx_t,\theta)$. Line 11 applies the triangle inequality to upperbound by the defined surrogate loss. Line 12 applies the assumed definition of $J(\theta)$. 

The intuition behind these steps is that difference between  the two distribution can be controlled via the surrogate loss on the expected supervisor. Thus, illustrating the closer the robot's policy matches the supervisor's policy on the supervisor's distribution, the smaller the total variational difference between the resulting two distributions will be. 
\end{proof}
  
One interesting result of Thereom 4.2 is that you can obtain a rate of $O(T\sqrt{T})$ without the assumption of convexity on the surrogate loss. In contrast DAgger's linear rate of $O(T)$ only applies to strongly convex surrogate loss function, which prohibits the analysis of decision trees and neural networks. 

\subsection{Finite Sample Analysis}
The above analysis demonstrates how the constant $T$ affects the bound in error. However, it is important to note that this is not the only variable that plays a role in performance. The size of the function class and number of demonstrations needed are are also important. 

Understanding how much data is needed to learn a function, is a well studied problem known as sample complexity analysis~\cite{anthony2009neural}. In this literature they use different metrics to describe the complexity of a function class and show rates on which a given function class would converge to the real solution. We refer the reader to \cite{vapnik2013nature}, for a review of such topics. 

An example of such results though is the following.  If we are interested in the total cost along a trajectory with respect to a policy, $\pi_{\theta}$, which is defined as $J(\theta) = \sum^T_{t=1} l(\pi_{\theta}(\bx_{t}),\pi_{\theta^*}(\bx_{t}))$.  If we assume the surrogate loss $l$ is convex with respect to $\theta$ and bounded between $[0,1]$, then if $\theta \in \Theta$. \\

\begin{theorem}\label{thm:sup}
For a policy, $\theta^N$ found via passive LfD, from $N$ trajectories collected from the supervisor the following is true with probability at least $1- 2\mbox{exp} (-m\delta^2/8)$

$$E_{p(\tau|\theta^*)} J(\theta^N)\leq T( 2R_{\Theta}(n) + \delta+ \frac{1}{\sqrt{n}})$$\\

\end{theorem}

In this result, $R_{\Theta}$ corresponds to the Rademacher complexity of the given function class $\Theta$. The Rademacher complexity is a measure of much a function can fit to random noise (i.e. no relation exists between $\bx_i$ and the label), more complex function classes can fit better and thus leading to slower rates of convergence.

Thus if your expert's policy is only a linear function, you may find the Rademacher term is small enough that $T$ is the dominant factor. However, if you are trying to learn a highly non-linear function, the Rademacher complexity could be much larger than $T$~\cite{vapnik2013nature} Thus, suggesting that only examining $T$ in the analysis can be misleading. 


\section{Experiments}

\begin{figure*}
\includegraphics{f_figs/var_grid.eps}
\caption{
    \footnotesize
Shown above is the normalized performance with respect to the expected supervisor, where $1.0$ indicates matching the expected supervisor's reward. The plots are averaged over 100 randomly generated 2D gridworld environments,  where the robot is taught to avoid penalty states and reach a goal state. Condition A examines when the robot's policy class is not able to learn the supervisors, which results in adaptivity leading to better performance. Condition D examines a larger robot policy class that contains the expected supervisor,and demonstrates negligible difference between adaptive and passive LfD.  to represent the supervisor. This leads to similar performance as Condition B, but requires more data. Finally, Condition C examines when the supervisor has noise added to the controls labels, this leads to more data being needed to converge to the expected supervisor, but the difference between passive and adaptive is still negligible.   }
\vspace*{-20pt}
\label{fig:var}
\end{figure*}

We provide experiments in a simulated GridWorld environment, which allows for us to vary the robot's policy class over a large number of randomly generated enviroments. Then we examine a linear dynamical system, which is used to show a limitation the ability of adaptive algorithms to converge. Finally, we perform human trials on 10 participants, who try to teach a robot how to singulate, or separate an object from a pile. 

\subsection{Varying Function Class}\label{sec:gdw}
In this experiment, we hypothesize that the performance gap of adaptive LfD diminishes as the the robot's policy class is increased. In a simulated GridWorld, we have a point robot that is trying to reach a goal state, at which it receives $+10$ reward. The robot receives $-10$ reward if it touches a penalty state, as shown in Fig. \ref{fig:grid_world}. The robot has a state space of $(x,y)$ coordinates and a set of actions consisting of $\lbrace$LEFT,RIGHT,FORWARD,BACKWARD NONE$\rbrace$, where NONE indicates staying at the current stop. For the transition dynamics $p(\bx_{t+1}|\bx_{t},\bu_t)$, the robot goes to an adjacent state that does not correspond to the intended control with probability $0.2$. 

We use Value Iteration to compute an optimal supervisor for our grid world domain. The optimal policy must learn to be robust to the noise in the dynamics, reach the goal state and then stay there. In all settings we provided the adaptive approach with one demonstration from the supervisor and then roll out its policy. 

Each plot shows the normalized performance where $1.0$ represents the expected performance of the value iteration supervisor. We run all trials over 100 randomly generated environments, which contain $X$ randomly selected penalty state and $1$ goal state. 


\noindent \textbf{Not in the Set} In Fig. \ref{fig:var} A, we show the case when the robot's policy class does not contain the supervisors, we used a Linear SVM for the policy class representation, which is commonly used in ~\cite{ross2010efficient,ross2010reduction,ross2013learning} . As illustrated the Adaptive DAgger approach does better than the Passive approach, which is a common result in the literature~\cite{ross2010efficient,ross2010reduction}.


Similar to prior work passive LfD  with a Linear SVM does worse than DAgger, which iteratively determines the distribution it policy induces and minimize the expected risk on it. However, because the robot's policy class did not contain the supervisor's it was not able to perform as well as the value iteration supervisor. 

\noindent \textbf{In the Set}
We next consider the situation where the robot's policy class contains the expected supervisor. We used decision trees with a depth of $100$ to obtain a complex enough function class to achieve this. 

As shown in Fig. \ref{fig:var}, DAgger and passive LfD both converge to the supervisor, but at the same rate. Thu, suggesting that advantage of using DAgger diminishes once the expected supervisor is in the robot's policy class. We note that because the robot's policy class obtains the supervisors convergance to the supervisor's performance is now obtained. 




\noindent \textbf{Noisy Supervisor}
We finally observe the effects of noise on the supervisor. Here we consider the case where noise is applied to observe label, thus the robot receive control labels that are $\bu = \pi_{\theta}(\bx) + \epsilon$,  where epsilon is an i.i.d distribution that selects a random control with probability $0.2$.

We use the  decision tree  of depth $100$ which due to its large function class is more susceptible to the noisy supervisor. We then compare the performance of passive versus adaptive LfD. As shown, both passive and adaptive are able to converge to the expected supervisor's normalized performance because they can average out the noise in the labels with enough data. Though it does take more data because passive LfD  and DAgger converge at a similar rate to the true expected supervisor. 




\subsection{Algorithmic Convergence }
In this experiment, we will demonstrate how adaptive techniques can fail to converge despite having the necessary function class to match the expert. 
We  consider the example where the robot needs to learn to get to a location in a 2D continuous world domain. The robot is represented as a point mass dynamics and the supervisor is computed using the infinite time horizon LQG, which results in a linear policy. 

The environment contains two sets of dynamics: 

$$x_{t+1} = A_1\bx_{t+1}+B_1\bu_t+w$$
$$x_{t+1} = A_2\bx_{t+1}+B_2\bu_t + w$$

where $w\sim \mathcal{N}(0,\sigma I)$. The dynamics for region B correspond to the controls being inverted for dynamics A. A target goal state and start state both lie in region A. 

The supervisor is a switching linear system, where each linear model is computed via the infinite horizon LQG for the specified dynamics. The robot's policy, $\pi_{\theta}$ is represented as a linear policy which is found via ridge regression.

We run passive Lfd and DAgger in this setting and plot the performance in Fig. \ref{fig:p_mass}. As shown, passive LfD is able to converge to the true supervisor's policy however the adaptivitiy of DAgger forces it to enter the region B of the work space and subsequently try to learn the two different supervisors. Thus, preventing it from converging. 

\begin{figure}
\centering
\includegraphics{f_figs/p_mass.eps}
\caption{
    \footnotesize
Left: A 2D workspace where a point mass robot is taught to go to the green circle starting from the blue circle. The world is divided into to two quadrants A and B, in B the controls are inverted in the dynamics thus resulting in $x=y$ and $y=x$. The supervisor is the infinite horizion LQG computed policy, which results in two different linear matrices in region A and B. Right: Illustrates how having a linear robot policy class can cause DAgger to fail  to converge due to it collecting data from region B.  }
\vspace*{-1pt}
\label{fig:p_mass}
\end{figure}

\subsection{Human Study for Planar Singulation}
We lastly perform a human study on a real robot, where people teach the robot to perform singulation task, or separate a object from a pile. Our hypothesis is that people will struggle providing retro-active feedback and cause the robot to try and learn more complex policies.  The objects used to form clutter are red extruded polygons.  For singulation task, we consider objects made of Medium Density Fiberboard with an average 4" diameter and 3" in height. 

The robot has a 2 dimensional internal state of base rotation and arm extension. The state space of the environment, $\mathcal{X}$, is captured by  an overhead Logitech C270 camera, which is positioned to capture the workspace that contains all cluttered objects and the robot arm. We use only the current image as state space, which captures positional information and a neural network architecture policy representation, the same as in~\cite{laskeyrobot}.

\begin{figure}
\centering
\includegraphics{f_figs/singulation.eps}
\caption{
    \footnotesize
Shown above is an example of an initial state (left) that the robot is presented with. It can vary in placement of the object in the pile, translation and rotation. A human is asked to singulate the object, which is to have the robot learn to push one object from the pile (right).  }

\label{fig:izzy_rw}
\end{figure}

The robot is commanded via position 
control with a  PID controller. Similar to \cite{laskeyshiv}, the controls, $\mathcal{U}$, to the robot are bounded changes in the internal state, which allows for us to easily register control signals to the demonstrations provided by supervisors as opposed to torque control. The control signals for each degree of freedom are continuous values with the following ranges: base rotation, $[-15^\circ,15^\circ]$, arm extension $[-1,1]$. The units are degrees for base rotation and centimeters for arm extension. 

During training and testing the initial state distribution $p(\bx_0)$ consists of sampling the relative pose of 4 objects from a distribution around their position in the pile and the pose of the pile as a whole. The pose of the pile is sampled from a Gaussian with variance \mlnote{get numbers}. Then using a virtual overlay,  a human is asked to place objects in their correct pose. 

The robot can be trained in either 2 ways DAgger or passive learning. In passive learning, the subject is asked to provide 60 demonstrations to the robot using an Xbox Controller. In active learning the user is first asked to provide 20 demonstrations via the Xbox Controller and then provides retro-active feedback for $K=2$ iterations of 20 demonstrations each. They used the same overlay technique as in ~\cite{laskeyrobot} to provide feedback. 

10 human subjects were selected who had a background in robotics research, but not in LfD. They were given a short demonstration of the learned robot policy and then asked to practice providing feedback through DAgger for 5 demonstrations and passive for 5 demonstrations. We then have each subject provide the first 20 demonstrations via passive feedback and then using counter-balancing to select whether they will perform passive or DAgger for the next 40 demonstrations.  

In Fig. \ref{fig:izzy_rw} , we show the average performance of the policies trained with DAgger and passive LfD. Each policy is evaluated on a hold out set of 30 initial states sampled from the same distribution as training. As shown, the policies learned with DAgger exhibit statistically significant worse performance than those with passive LfD. Thus, sugggesting that DAgger could perform worse than passive Lfd on a task with actual human supervision. 

\noindent \textbf{Post Analysis}
\mlnote{updating this with numbers from post analysis soon}
In order to gain some insight into the human study, we compared the equivalence of human retroactive feedback and tele-operation. We designed a  experiment: the human was to tele-operate the robot five times and asked to provide retro-active feedback via the DAgger interface to try and match the controls given by tele-operation.    
The controls of the human tele-operation with those given by retroactive feedback. We compared the controls via first normalizing them between $[-1,1]$ based on their bounds an then used squared euclidean distance of the rotation and translation control. The same loss used in training of the policy. 

This measurement gives a rough percentage difference between the human's teleoperation actions, and their retroactive feedback. A more intuitive physical measurement of the differences was also performed. To do this, we computed the resulting robot state given that it moved the full distance of the applied delta, then finding the Euclidian distance between the subsequent state of robot, and the state predicted from the delta applied from the retroactive feedback. This gives units of centimeters to the error per delta.

\mlnote{Need to better explain this} We performed this experiment with five subjects, and observed an average normalized distance of $0.44$, or $44\%$ deviation. By comparison, the normalized distance on the test set during training of the policy gave a normalized error of about $22\%$. The average euclidian distance of the states was $0.8$ cm. These results indicate a  difference between the controls applied by retroactive feedback and those applied by the human performing teleoperation.

Thus suggesting that the intention of humans providing feedback may be lost in the labeling interface. We acknowledge that this might be remedy by providing a more intuitive labeling interface, however a trade-off exists between time spent designing an interface for a given task and collecting more data via passive LfD. 

\begin{figure}
\centering
\includegraphics{f_figs/izzy_reward.eps}
\caption{
    \footnotesize
Shown is averaged success at the singulation task over the 10 human subjects. Each policy is evaulated 30 times on the a held out set of test configurations. The first 20 rollouts are from the supervisor rolling out there policy and the next 40 are collected via retro-active feedback for adaptive and tele-operated demonstrations for passive. Passive LfD shows a 20$\%$ improvement in success at the end. }
\vspace*{-20pt}
\label{fig:izzy_rw}
\end{figure}



\section{Conclusion and Future Work}
We thus conclude our analysis on the trade-offs between passive and adaptive Lfd approaches. We demonstrated that if a robot's function class contains the expected supervisor's policy, then the  performance difference between adaptive algorithms, such as DAgger,  and passive Lfd techniques can become negligible.  Furthermore, we also provided examples where adaptivity fails to converge despite being able to represent the supervisor. Thus illustrating a short-coming in the regret style analysis for DAgger. 

Our final experiment in having humans teach a robot to singulate objects suggest that adaptivity can lead to worse performance. Our post analysis seems to suggest that humans had a harder time providing retro-active feedback via a labeling interface. While, this may be overcome through a better labeling interface, an inherent problem persists in the inability of a human to observe how well their controls would have performed. 

Finally, we offered a new theoretical  way to analysis passive Lfd and demonstrate that the depedence on the time horizon constant, $T$, can be $T\sqrt{T}$ and not $T^2$, a result that does not need the assumption of strongly convexity.  For convex loses, we are able to offer data-dependent sample complexity bounds that illustrate how the function class of the learning algorithm and number of demonstrations effect this bound on error. 

The data-dependent results demonstrate that in the worst-case a heavy tail could exits in terms of the number of demonstrations needed (i.e. a large number of demonstrations could be need). Thus, presenting a challenge in obtaining an industrial level of reliability in unstructured domains. However, in future work we are interested in examining ways around this limitation. Below are two possible ways:

\noindent \textbf{Active Learning} Uniform sampling from the initial state distribution $p(\bx_0)$, may not be an optimal approach to learning the supervisor's policy. In active learning, an algorithm would try to select more informative initial states for the robot to learn from. An example of this is the field of optimal design, where measurments are collected in a way to reduce variance in the label and lead to more robust data analysis. Extending this to LfD would be an interesting oppurtunity. 

\noindent \textbf{Synthetic Data Generetion} Another approach to reduce data-dependence is to synthetically grow your dataset. For example, consider a robot learning how to grasp an isolated object  with image data as the state space. One could take the given image state and translate both the object and grasp label to create more demonstrations. A similar approach has been used recently in self-driving cars. However, a formal approach to checking when synthetic data generation is possible would be an exciting new direction. 

\bibliographystyle{IEEEtranS}
\bibliography{references}

\appendix
\subsection{Notation and Problem Statement}
Given a discrete-time stochastic dynamical system over the state-space $\mathcal{X}$ and control-space $\mathcal{U}$
\begin{equation}
x_{t+1} = f(x_t,u_t) + \epsilon_t~~x_0\sim p(x_0)~~\epsilon_t \sim \text{i.i.d}, \label{sde}
\end{equation}
both passive and adaptive LfD model the supervisor as a stochastic state-feedback control policy $\pi^*: \mathcal{X} \mapsto \mathcal{U}$:
\[
u_{t} \sim \pi^*(x_t).
\]
We assume that this distribution is parametric, parametrized by $\theta^* \in \Theta$, and where $\Theta$ denotes the parameter set over all possible policies--we will use $\pi_{\theta}$ to denote a given policy.  
Every policy parametrized by has a state distribution at time $t$:
\[x_t \sim \theta =  p(x_0) \prod^t_{i=1} p(x_{i+1}|x_{i},u_{i})p(u_i|x_i,\theta).\]
\sknote{how do you do retroactive feedback stochastically?}

Given independent and identically distributed episodic observations (each of a fixed-length $T$) of Equation \ref{sde}:
\[
D = \{d_1,...,d_N\}~~~~d_i = [(x_1,u_1),...,(x_T,u_T)],
\]
the LfD problem is to find a parametrized policy $\pi_{\theta}$ that minimizes the expected disagreement $\pi_{\theta^*}$ over $T$ time-steps of the system:
\begin{equation}
\min_{\theta \in \Theta} \sum_{t=0}^T \mathbf{E}_{x_t \sim \theta} [ \ell(\pi_{\theta}(x_t), \pi_{\theta^*}(x_t))]. \label{obj}
\end{equation}

\subsection{Empirical Risk Minimizer (Passive LfD)}
One approach to this problem is to find the $\hat{\theta}$ that minimizes the loss over each state seen in the demonstrations:
\[
\min_{\theta \in \Theta} \sum_{d=0}^N \sum_{t=0}^T \ell(\pi_{\theta}(x_{d,t}), u_{d,t}),
\]
and one can re-write this minimization problem as:
\begin{equation}
\hat{\theta} = \arg\min_{\theta \in \Theta} \sum_{d=0}^N  J(\theta),\label{erm}
\end{equation}
where $J(\theta) = \sum_{t=0}^T \ell(\pi_{\theta}(x_{d,t}), u_{d,t})$.
Equation \ref{erm} is called the empirical risk minimizer (ERM) as it estimates the true risk--the disagreement with the expected supervisor over the sample-space of demonstrations:
\[
\min_{\theta \in \Theta} \mathbf{E}_{d \sim D^\infty}  J(\theta).
\]
An important notion in ERM is consistency, i.e., $\hat{\theta} \rightarrow \theta^* \text{ a.s}$. This can be achieved for certain classes of functions via a uniform convergence argument \sknote{Someone who knows this advise--and replace consistency with whatever term you want}.
Our analysis assumes consistency and relaxing this assumption is a natural next step in future work.

\vspace{0.5em}\noindent \textbf{Remark: } Even with the fairly strong assumption of consistency, the LfD problem remains to be complex. The empirical risk minimization problem is measuring the expected risk over the distribution of states seen by the supervisor's policy. This is not quite the same as measuring the expected error over T time-steps of execution in Equation \ref{obj}. For any finite sample, the empirical risk minimizer will in general be $\hat{\theta} \ne \theta^*$, and the distribution of states visited by applying $\pi_{\hat{\theta}}$ might be different than the distribution of states visited by applying $\pi_{\theta^*}$.

\subsection{Follow-The-Leader (DAgger)}
To address the mismatch between supervisor and execution distributions, Ross et al. proposed an algorithm (DAgger) derived from online optimization rather than supervised learning. In online optimization, one usually considers a scenario where a sequence of data arrives in an unknown (and possible adversarial) stochastic process.
DAgger collects an initial set of $N'$ demonstrations, and computes the empirical risk minimizer $\hat{\theta}$.
Then, it executes this policy and collects samples of states visited by $\pi_{\hat{\theta}}$.
It appends these samples to the initial set of demonstrations and re-optimizes.
This approach is a form of online optimization called Follow-The-Leader, i.e., apply most successful, or ``leading'', parameter over the previous observations. 

\vspace{0.5em}\noindent \textbf{Remark: } Ross et al. proved a error bound with DAgger that showed error with DAgger is $\mathcal{O}(T\epsilon_n)$ where $\epsilon_n = \mathbf{E}_{d \sim D^\infty}  J(\hat{\theta})$. It is important to differentiate this bound from the hypothetical ``best policy in the class'', which would have $\epsilon = \mathbf{E}_{d \sim D^\infty}  J(\theta^*)$ and $\epsilon \le \epsilon_n$. In this sense, DAgger ensures that the error rate grows linearly with the best policy seen in the current dataset, but does not guarantee it will visit the same states (in distribution) as the supervisor as $N \rightarrow \infty$.







































\end{document}